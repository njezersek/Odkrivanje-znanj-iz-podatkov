\documentclass[10pt,a4paper]{article}
\usepackage[utf8x]{inputenc}

\usepackage[utf8x]{inputenc}

\usepackage{fancyhdr}

\usepackage[pdftex]{graphicx} % Required for including pictures
\usepackage[pdftex,linkcolor=black,pdfborder={0 0 0}]{hyperref} % Format links for pdf
\usepackage{calc} % To reset the counter in the document after title page
\usepackage{enumitem} % Includes lists

\usepackage[normalem]{ulem}

\usepackage{textcomp}
\usepackage{eurosym}

\usepackage{xcolor}
\usepackage{graphicx}
\graphicspath{ {./images/} }

\usepackage{ amssymb } % extra math symbols
\usepackage{ amsmath } % extra math symbols
\usepackage{amsthm}
\usepackage{ dsfont } % font za množice
% tabele
\usepackage{array}
\usepackage{wrapfig}
\usepackage{multirow}
\usepackage{tabularx}
\usepackage{multicol}
\usepackage{listings}
\usepackage{caption}
\usepackage{tikz,forest}
\usetikzlibrary{arrows.meta}

\newtheorem*{trditev}{Trditev}
\newtheorem*{izrek}{Izrek}
\newtheorem*{posledica}{Posledica}
\newtheorem*{definicija}{Definicija}

\frenchspacing % No double spacing between sentences
\setlength{\parindent}{0pt}
\setlength{\parskip}{0.5em}

\usepackage{mathtools}
\usepackage{blkarray, bigstrut} %

%\pagenumbering{gobble}


\DeclareCaptionLabelFormat{algocaption}{Algorithm \thenalg} % defines a new caption label as Algorithm x.y

\lstnewenvironment{algorithm}[1][] %defines the algorithm listing environment
{   
    %\refstepcounter{nalg} %increments algorithm number
    %\captionsetup{labelformat=algocaption,labelsep=colon} %defines the caption setup for: it ises label format as the declared caption label above and makes label and caption text to be separated by a ':'
    \lstset{ %this is the stype
        mathescape=true,
        %frame=tB,
        %numbers=left, 
        numberstyle=\tiny,
        basicstyle=\scriptsize, 
        keywordstyle=\color{black}\bfseries\em,
        keywords={,vhod, izhod, vrni, dokler, izvajaj, konec, zanke} %add the keywords you want, or load a language as Rubens explains in his comment above.
        %numbers=left,
        %xleftmargin=.04\textwidth,
        %#1 % this is to add specific settings to an usage of this environment (for instnce, the caption and referable label)
    }
}
{}
\usepackage{layouts}
\usepackage{titlesec} 

\definecolor{codegreen}{rgb}{0,0.6,0}
\definecolor{codegray}{rgb}{0.5,0.5,0.5}
\definecolor{codepurple}{rgb}{0.58,0,0.82}
\definecolor{backcolour}{rgb}{0.95,0.95,0.92}

\lstdefinestyle{mystyle}{
	basicstyle=\small, 
	language=python, 
	breaklines, tabsize=2, 
	frame=leftline,xleftmargin=10pt,xrightmargin=10pt,framesep=10pt
}

\pagestyle{fancy}
\fancyhf{}
\rhead{Jernej Jezeršek}
\lhead{Uvod v odkrivanje znanj iz podatkov}
\rfoot{\thepage}


\begin{document}

{\huge Domača naloga 5} \\
{\Huge \textbf{Multinomska logistična \\regresija}} 

\section*{Pravilnost implementacije}
Pri implementaciji sem si pomagal z zapiski Softmax Regression\footnote{\url{http://deeplearning.stanford.edu/tutorial/supervised/SoftmaxRegression/}}.
Vse formule sem pretvoril v matrično obliko (edina \texttt{for} zanka je pri prečnem preverjanju).

\subsection*{Funkcija \texttt{softmax}}
Poglejmo, da se pravilno obnaša na preprostem primeru.
\begin{align*}
	X &= \begin{bmatrix}
		1 & 0 \\
		0 & 1
	\end{bmatrix}
	&
	\theta = \begin{bmatrix}
		10 & 0 & 0 \\
		0 & 1 & 1
	\end{bmatrix}
\end{align*}

Vrstice matrike $X$ so primerki, stolpci pa atributi. Stolpci matrike $\theta$ pripadajo razredom, vrstice pa atributom.

Iz matrike $\theta$ se vidi, da prvi atribut prispeva k večji verjetnosti za 0. razred, drugi atribut pa k večji verjetnosti za
1. in 2. razred.

Rezultat funkcije \texttt{softmax} za prvo vrstico $X$ bi moral imeti veliko verjetnost za 0. razred in maljhno za preostala dva.
Rezultat za drugo vrstico pa bi moral biti: majhna verjetnost za 0. razred in enaka verjetnost za 1. in 2. rezred.

Funkcijo sem testiral na takem in podobnih primerih, in se obnaša po po pričakovanju opisanem zgoraj.

\subsection*{Funkcija \texttt{cost}}
Ta funkcija bi morala vrniti vrednost blizu 0, če se napoved \texttt{softmax}-a ujema z oznakami $y$, sicer pa
veliko negarivno vrednost.

Uporabimo isti primer kot prej in poženemo funkcijo \texttt{cost} z različnimi verdnosti za $y$ in dobimo:
\begin{center}
	\renewcommand{\arraystretch}{1.5}
	\begin{tabular}{c | c  c }
		$y$ & \texttt{softmax}\\ \hline
		$\begin{bmatrix} 0 & 1 \end{bmatrix}$ & -0.6932 \\
		$\begin{bmatrix} 0 & 2 \end{bmatrix}$ & -0.6932 \\
		$\begin{bmatrix} 1 & 0 \end{bmatrix}$ & -20.6933 \\
	\end{tabular}
\end{center}

Ti rezultati so smiselni, saj ima privi primerek največjo verjetnost za razred 0, drugi pa enoako verjetnost za razred 1 ali 2.


\subsection*{Funkcija \texttt{grad}}
Spet uporabimo isti primer, tokrat za $y$ izberemo $[1 \ 0]$ za katerega je $\theta$ najslabše prilagojena.

Gradient (zaokrožen) po posameznih elementih $\theta$ je:
\[ \begin{bmatrix}
	-1 &  1 & 0 \\
	1 & -0.5 & -0.5
\end{bmatrix}\]

Za dani $X$ in $y$ bi morala biti $\theta$ čim bližje:
\[ \begin{bmatrix}
	0 &  1 & 0 \\
	1 & 0 & 0
\end{bmatrix}\]

Če trenutni $\theta$ prištejemo gradient, bo bližje idealni vrednosti. Zato je rezultat smiselen.

\subsection*{Regularizacija}
Pri povečevanju vpliva regularizacije ($\lambda$), vrednost \texttt{log loss} pri prečnem preverjanju na učnih podatkih počasi pada nato 
pa spet naraste, kar je v skladu s pričakovanji.

\section*{Opis reševanja napovedi}
Podatke (\texttt{train.csv}, \texttt{test.csv}) sem s knjižnico \texttt{Pandas} in jim odstranil stolpec id. 
V tabeli učnih podatkov sem stolpec \texttt{target} spremenil v števila od $0$ do $8$. Nato sem podatke pretvoril v \texttt{numpy} sezname.

Na učnih podatkih sem izvedel prečno preverjanje ($k=5$) in dobil naslednje rezultate:
\begin{center}
	\begin{tabular}{c | c  c }
		$\lambda$ & accuracy & log loss\\ \hline
		0.0 & 0.762800 & 0.645148 \\
		0.1 & 0.762700 & 0.645068 \\
		0.2 & 0.762960 & 0.644904 \\
		0.4 & 0.762500 & 0.644552 \\
		0.8 & 0.762340 & 0.644525 \\
		1.6 & 0.762140 & 0.644292 \\
		3.2 & 0.762140 & 0.644163 \\
		6.4 & 0.761680 & 0.644251
	\end{tabular}
\end{center}

Glede na te rezultate sem za regularizacijo uporabil $\lambda = 3.2$.

Za končno napoved sem model naučil na vseh učnih podatkih. Napoved modela sem shranil v \texttt{final.txt}.

Program za gradnjo končnega modela porabi \textbf{9 s}.


\end{document}
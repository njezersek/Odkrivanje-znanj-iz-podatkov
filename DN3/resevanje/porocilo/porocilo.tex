\documentclass[11pt,a4paper]{article}
\usepackage[utf8x]{inputenc}

\usepackage[utf8x]{inputenc}

\usepackage{fancyhdr}

\usepackage[pdftex]{graphicx} % Required for including pictures
\usepackage[pdftex,linkcolor=black,pdfborder={0 0 0}]{hyperref} % Format links for pdf
\usepackage{calc} % To reset the counter in the document after title page
\usepackage{enumitem} % Includes lists

\usepackage[normalem]{ulem}

\usepackage{textcomp}
\usepackage{eurosym}

\usepackage{xcolor}
\usepackage{graphicx}
\graphicspath{ {./images/} }

\usepackage{ amssymb } % extra math symbols
\usepackage{ amsmath } % extra math symbols
\usepackage{amsthm}
\usepackage{ dsfont } % font za množice
% tabele
\usepackage{array}
\usepackage{wrapfig}
\usepackage{multirow}
\usepackage{tabularx}
\usepackage{multicol}
\usepackage{listings}
\usepackage{caption}
\usepackage{tikz,forest}
\usetikzlibrary{arrows.meta}

\newtheorem*{trditev}{Trditev}
\newtheorem*{izrek}{Izrek}
\newtheorem*{posledica}{Posledica}
\newtheorem*{definicija}{Definicija}

\frenchspacing % No double spacing between sentences
\setlength{\parindent}{0pt}
\setlength{\parskip}{0.5em}

\usepackage{mathtools}
\usepackage{blkarray, bigstrut} %

%\pagenumbering{gobble}


\DeclareCaptionLabelFormat{algocaption}{Algorithm \thenalg} % defines a new caption label as Algorithm x.y

\lstnewenvironment{algorithm}[1][] %defines the algorithm listing environment
{   
    %\refstepcounter{nalg} %increments algorithm number
    %\captionsetup{labelformat=algocaption,labelsep=colon} %defines the caption setup for: it ises label format as the declared caption label above and makes label and caption text to be separated by a ':'
    \lstset{ %this is the stype
        mathescape=true,
        %frame=tB,
        %numbers=left, 
        numberstyle=\tiny,
        basicstyle=\scriptsize, 
        keywordstyle=\color{black}\bfseries\em,
        keywords={,vhod, izhod, vrni, dokler, izvajaj, konec, zanke} %add the keywords you want, or load a language as Rubens explains in his comment above.
        %numbers=left,
        %xleftmargin=.04\textwidth,
        %#1 % this is to add specific settings to an usage of this environment (for instnce, the caption and referable label)
    }
}
{}
\usepackage{layouts}
\usepackage{titlesec} 

\definecolor{codegreen}{rgb}{0,0.6,0}
\definecolor{codegray}{rgb}{0.5,0.5,0.5}
\definecolor{codepurple}{rgb}{0.58,0,0.82}
\definecolor{backcolour}{rgb}{0.95,0.95,0.92}

\lstdefinestyle{mystyle}{
	basicstyle=\small, 
	language=python, 
	breaklines, tabsize=2, 
	frame=leftline,xleftmargin=10pt,xrightmargin=10pt,framesep=10pt
}

\pagestyle{fancy}
\fancyhf{}
\rhead{Jernej Jezeršek}
\lhead{Uvod v odkrivanje znanj iz podatkov}
\rfoot{\thepage}


\begin{document}

{\huge Domača naloga 3} \\
{\Huge \textbf{Napovedovanje prihodov\\ avtobusov LPP}} \\

\section*{Predtekmovanje}
Vsako vožnjo sem pospisal z naslednjimi značilkami:
\begin{itemize}
	\item \textbf{Dan v tednu} - binarno, 7 značilk
	\item \textbf{Ura v dnevu} - binarno, 24 značilk
	\item \textbf{Voznik} - binarno, ena značilka za vsakega voznika
	\item \textbf{Praznik} - binarno, ena značilka
	\item \textbf{Šolske počitnice} - binarno, ena značilka
	\item \textbf{Temperature pod ničlo} - binarno ena značilka
	\item \textbf{Dan v letu} - polinomska značilka 2. stopnje
	\item \textbf{Minute v dnevu} - polinomska značilka 4. stopnje
	\item \textbf{Temperature, pdavine, veter} - polinomska znančilka 2. stopnje
	\item \textbf{Dan v tednu} - polinomska značilk 2. stopnje
\end{itemize}

Za vsako voznjo sem izračunal trajanje v minutah, ki ga model poskuša napovedati.

\subsubsection*{Rezultati}
\begin{center}
	\begin{tabular}{c | c | c}
		\textit{program} & \textit{datoteka z napovedjo} & \textit{rezultat} \\ \hline
		\texttt{predtekmovanje.py} & \texttt{result7.txt} & 120.13918
	\end{tabular}
\end{center}
\section*{Tekmovanje}

\subsubsection*{Prvi pristop:}
Najprej sem poskusil z metodo iz predtekmovanja obravnavati vsako linijo posebej, a
ta pristop se ni posebej dobro obnesel (verjetno zaradi prevelikega števila značilk) z rezultatom, ki je slabši od povprečja (306.89).

Zato sem ohranil le naslednje značilke:
\begin{itemize}
	\item \textbf{Dan v tednu} - binarno, 7 značilk
	\item \textbf{Ura v dnevu} - binarno, 24 značilk
	\item \textbf{Voznik} - binarno, ena značilka za vsakega voznika
	\item \textbf{Šolske počitnice} - binarno, ena značilka
	\item \textbf{Prosti dan (praznik ali konec tedna)} - binarno, ena značilka
	\item \textbf{Temperature pod ničlo} - binarno ena značilka
	\item \textbf{Dež (količina padavin $>$ 1 mm)} - binarno ena značilka
	\item \textbf{Veter (hitrost $>$ 1.5 m/s)} - binarno ena značilka
	\item \textbf{Sneg (dež in temperature pod ničlo)} - binarno ena značilka
\end{itemize}

\subsubsection*{Drugi pristop:}
Tokrat sem naredil le en model. Ideja je, da se en model lahko uči na vseh vožnjah in
skuša napovedati relativno zamudo glede na povprečno trajanje vožnje.

Najprej sem za vsako linijo izračunal mediano trajanja. Nato sem za vsako vožnjo
izračunal faktor odstopanja od povprečja $k = t / \overline{t}$, kjer je $t$ dejansko trajanje, $\overline{t}$ pa mediana trajanja za linijo.

Model skuša napovedati faktor odstopanja $k$.

Model uporablja iste značilke, kot prejšnji.

\subsubsection*{Rezultati}
\begin{center}
	\begin{tabular}{c | c | c}
		\textit{program} & \textit{datoteka z napovedjo} & \textit{rezultat na lestvici} \\ \hline
		\texttt{tekmovanje1.py} & \texttt{tesult\_t\_4.txt} & 176.86339 \\
		\texttt{tekmovanje2.py} & \texttt{tesult\_t\_2\_2.txt} & 184.61265
	\end{tabular}
\end{center}

\end{document}
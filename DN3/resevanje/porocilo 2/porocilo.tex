\documentclass[11pt,a4paper]{article}
\usepackage[utf8x]{inputenc}

\usepackage[utf8x]{inputenc}

\usepackage{fancyhdr}

\usepackage[pdftex]{graphicx} % Required for including pictures
\usepackage[pdftex,linkcolor=black,pdfborder={0 0 0}]{hyperref} % Format links for pdf
\usepackage{calc} % To reset the counter in the document after title page
\usepackage{enumitem} % Includes lists

\usepackage[normalem]{ulem}

\usepackage{textcomp}
\usepackage{eurosym}

\usepackage{xcolor}
\usepackage{graphicx}
\graphicspath{ {./images/} }

\usepackage{ amssymb } % extra math symbols
\usepackage{ amsmath } % extra math symbols
\usepackage{amsthm}
\usepackage{ dsfont } % font za množice
% tabele
\usepackage{array}
\usepackage{wrapfig}
\usepackage{multirow}
\usepackage{tabularx}
\usepackage{multicol}
\usepackage{listings}
\usepackage{caption}
\usepackage{tikz,forest}
\usetikzlibrary{arrows.meta}

\newtheorem*{trditev}{Trditev}
\newtheorem*{izrek}{Izrek}
\newtheorem*{posledica}{Posledica}
\newtheorem*{definicija}{Definicija}

\frenchspacing % No double spacing between sentences
\setlength{\parindent}{0pt}
\setlength{\parskip}{0.5em}

\usepackage{mathtools}
\usepackage{blkarray, bigstrut} %

%\pagenumbering{gobble}


\DeclareCaptionLabelFormat{algocaption}{Algorithm \thenalg} % defines a new caption label as Algorithm x.y

\lstnewenvironment{algorithm}[1][] %defines the algorithm listing environment
{   
    %\refstepcounter{nalg} %increments algorithm number
    %\captionsetup{labelformat=algocaption,labelsep=colon} %defines the caption setup for: it ises label format as the declared caption label above and makes label and caption text to be separated by a ':'
    \lstset{ %this is the stype
        mathescape=true,
        %frame=tB,
        %numbers=left, 
        numberstyle=\tiny,
        basicstyle=\scriptsize, 
        keywordstyle=\color{black}\bfseries\em,
        keywords={,vhod, izhod, vrni, dokler, izvajaj, konec, zanke} %add the keywords you want, or load a language as Rubens explains in his comment above.
        %numbers=left,
        %xleftmargin=.04\textwidth,
        %#1 % this is to add specific settings to an usage of this environment (for instnce, the caption and referable label)
    }
}
{}
\usepackage{layouts}
\usepackage{titlesec} 

\definecolor{codegreen}{rgb}{0,0.6,0}
\definecolor{codegray}{rgb}{0.5,0.5,0.5}
\definecolor{codepurple}{rgb}{0.58,0,0.82}
\definecolor{backcolour}{rgb}{0.95,0.95,0.92}

\lstdefinestyle{mystyle}{
	basicstyle=\small, 
	language=python, 
	breaklines, tabsize=2, 
	frame=leftline,xleftmargin=10pt,xrightmargin=10pt,framesep=10pt
}

\pagestyle{fancy}
\fancyhf{}
\rhead{Jernej Jezeršek}
\lhead{Uvod v odkrivanje znanj iz podatkov}
\rfoot{\thepage}


\begin{document}

{\huge Domača naloga 4} \\
{\Huge \textbf{Avtobusi LPP s poljubnimi\\ metodami}} \\

\section*{Tekmovanje}

Preiskusil sem različne modele iz knjižnice \textit{Scykit Learn}. Najboljše rezultate sem dosegel z \texttt{RandomForestRegressor}.

Za vsako linijo sem naredil en model z naslednjimi atributi:
\begin{itemize}
	\item \textbf{Voznik} - celo število
	\item \textbf{Dan v tednu} - celo število
	\item \textbf{Mesec} - celo število
	\item \textbf{Temperature pod ničlo} - binarno ena značilka
	\item \textbf{Šolske počitnice} - binarno, ena značilka
	\item \textbf{Prosti dan (praznik ali konec tedna)} - binarno, ena značilka
	\item \textbf{Dež (količina padavin v mm)} - število
	\item \textbf{Sekunda v dnevu} - zakodirano kot 2d točka na enotski krožnici, kjer kot predstavlja čas \textit{(na ta način se ohrani podobnost med začetkom in koncem dneva)}
	\item \textbf{Dan v letu} - zakodirano kot 2d točka na krožnici
\end{itemize}


\subsubsection*{Rezultati}
\begin{center}
	\begin{tabular}{c | c | c}
		\textit{program} & \textit{datoteka z napovedjo} & \textit{rezultat na lestvici} \\ \hline
		\texttt{tekmovanje.py} & \texttt{tesult\_t3\_7.txt} & 156.95223 
	\end{tabular}
\end{center}

\end{document}